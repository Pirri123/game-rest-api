%%%%%%%%%%%%  Generated using docx2latex.com  %%%%%%%%%%%%%%

%%%%%%%%%%%%  v2.0.0-beta  %%%%%%%%%%%%%%

\documentclass[12pt]{article}
\usepackage{amsmath}
\usepackage{latexsym}
\usepackage{amsfonts}
\usepackage[normalem]{ulem}
\usepackage{array}
\usepackage{amssymb}
\usepackage{graphicx}
\usepackage[backend=biber,
style=numeric,
sorting=none,
isbn=false,
doi=false,
url=false,
]{biblatex}\addbibresource{bibliography.bib}

\usepackage{subfig}
\usepackage{wrapfig}
\usepackage{wasysym}
\usepackage{enumitem}
\usepackage{adjustbox}
\usepackage{ragged2e}
\usepackage[svgnames,table]{xcolor}
\usepackage{tikz}
\usepackage{longtable}
\usepackage{changepage}
\usepackage{setspace}
\usepackage{hhline}
\usepackage{multicol}
\usepackage{tabto}
\usepackage{float}
\usepackage{multirow}
\usepackage{makecell}
\usepackage{fancyhdr}
\usepackage[toc,page]{appendix}
\usepackage[hidelinks]{hyperref}
\usetikzlibrary{shapes.symbols,shapes.geometric,shadows,arrows.meta}
\tikzset{>={Latex[width=1.5mm,length=2mm]}}
\usepackage{flowchart}\usepackage[paperheight=11.0in,paperwidth=8.5in,left=1.0in,right=1.0in,top=1.0in,bottom=1.0in,headheight=1in]{geometry}
\usepackage[utf8]{inputenc}
\usepackage[T1]{fontenc}
\TabPositions{0.5in,1.0in,1.5in,2.0in,2.5in,3.0in,3.5in,4.0in,4.5in,5.0in,5.5in,6.0in,}

\urlstyle{same}


 %%%%%%%%%%%%  Set Depths for Sections  %%%%%%%%%%%%%%

% 1) Section
% 1.1) SubSection
% 1.1.1) SubSubSection
% 1.1.1.1) Paragraph
% 1.1.1.1.1) Subparagraph


\setcounter{tocdepth}{5}
\setcounter{secnumdepth}{5}


 %%%%%%%%%%%%  Set Depths for Nested Lists created by \begin{enumerate}  %%%%%%%%%%%%%%


\setlistdepth{9}
\renewlist{enumerate}{enumerate}{9}
		\setlist[enumerate,1]{label=\arabic*)}
		\setlist[enumerate,2]{label=\alph*)}
		\setlist[enumerate,3]{label=(\roman*)}
		\setlist[enumerate,4]{label=(\arabic*)}
		\setlist[enumerate,5]{label=(\Alph*)}
		\setlist[enumerate,6]{label=(\Roman*)}
		\setlist[enumerate,7]{label=\arabic*}
		\setlist[enumerate,8]{label=\alph*}
		\setlist[enumerate,9]{label=\roman*}

\renewlist{itemize}{itemize}{9}
		\setlist[itemize]{label=$\cdot$}
		\setlist[itemize,1]{label=\textbullet}
		\setlist[itemize,2]{label=$\circ$}
		\setlist[itemize,3]{label=$\ast$}
		\setlist[itemize,4]{label=$\dagger$}
		\setlist[itemize,5]{label=$\triangleright$}
		\setlist[itemize,6]{label=$\bigstar$}
		\setlist[itemize,7]{label=$\blacklozenge$}
		\setlist[itemize,8]{label=$\prime$}

\setlength{\topsep}{0pt}\setlength{\parskip}{8.04pt}
\setlength{\parindent}{0pt}

 %%%%%%%%%%%%  This sets linespacing (verticle gap between Lines) Default=1 %%%%%%%%%%%%%%


\renewcommand{\arraystretch}{1.3}


%%%%%%%%%%%%%%%%%%%% Document code starts here %%%%%%%%%%%%%%%%%%%%



\begin{document}
Ernesto Ramírez A00513925\par

Resumen Modelos Àgiles\par

El modelo ágil es una herramienta muy poderosa para la planeación de proyectos de software, hardware, comunicación y redes, instalación de hardware, auditoría, entre otros.\par

Dentro del modelos ágiles hay tres fases necesarias, primero, la planeación, en la cual se consideran los elementos necesarios para el proyecto, y los recursos que se necesitan para lograrlo. Posteriormente la siguiente fase es la de ejecución, en la cual se pone en marcha lo planeado en la etapa anterior. Por último, está la etapa de soporte, en la cual se le da mantenimiento a lo entregado en la etapa anterior, penara evitar fallas y optimizar el servicio.\par

\setlength{\parskip}{0.0pt}
El manifestó ágil es un documento que recopila la filosofía y los principios detrás de este paradigma de desarrollo, sus puntos clave son: Valoración del individuo y las interacciones del equipo de desarrollo sobre el proceso y las herramientas. Desarrollar software que funcione más que la documentación del mismo. La colaboración con el cliente más que la negociación de su contrato. Responde a los cambios más que seguir con el plan establecido.\par


\vspace{\baselineskip}
A continuación, se describirán las principales metodologías ágiles:\par


\vspace{\baselineskip}
Scrum se enfoca en un desarrollo iterativo e incremental, además de querer que el proceso de desarrollo sea transparente y fácil de adaptar. En general se trata de dividir la carga de trabajo en bloques con duración de un par de semanas llamadas sprints.\par


\vspace{\baselineskip}
Existen diferentes roles dentro de un proyecto de Scrum, el Scrum Master, quien se encargará de liderar al equipo y asegurar de que se cumplan las metas de los sprints y en general de que se adopte Scrum correctamente. Está el Product Owner, quien se enfoca en una visión de alto nivel, orientada al negocio, están los Stakeholder, quienes son personas cuyos intereses son importantes para el proyecto, pero no desarrollan nada en sí. Los usuarios, quienes son los que al final interactuarán con el producto, y el equipo de desarrollo, quienes se encargan de programar el producto final.\par


\vspace{\baselineskip}
En general, los métodos ágiles son las más comunes en la actualidad y son esenciales para el desarrollo moderno del software, por ende, no se deben de ignorar. \par


\vspace{\baselineskip}

\vspace{\baselineskip}

\vspace{\baselineskip}

\vspace{\baselineskip}

\vspace{\baselineskip}

\vspace{\baselineskip}

\vspace{\baselineskip}

\vspace{\baselineskip}

\vspace{\baselineskip}

\vspace{\baselineskip}

\vspace{\baselineskip}

\vspace{\baselineskip}

\vspace{\baselineskip}

\vspace{\baselineskip}

\vspace{\baselineskip}

\printbibliography
\end{document}